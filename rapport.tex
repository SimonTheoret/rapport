%%
%% This is file `gabaritmem.tex',
%% generated with the docstrip utility.
%%
%% The original source files were:
%%
%% dms.dtx  (with options: `memoire,gabarit')
%% Example TeX file for the documentation
%% of the jurabib package
%% Copyright (C) 1999, 2000, 2001 Jens Berger
%% See dms.ins  for the copyright details.
%%
%%% ====================================================================
%%%  @LaTeX-file{
%%%     filename        = "dms.dtx",
%%%     author    = "Nicolas Beauchemin, Damien Rioux-Lavoie, Victor Fardel, Jonathan Godin",
%%%     copyright = "Copyright (C) 2000 , DMS
%%%                  all rights reserved.  Copying of this file is
%%%                  authorized only if either:
%%%                  (1) you make absolutely no changes to your copy,
%%%                  including name; OR
%%%                  (2) if you do make changes, you first rename it
%%%                  to some other name.",
%%%     address   = "Département de Mathématiques et de Statistique",
%%%     telephone = "514-343-6705",
%%%     FAX       = "514-343-5700",
%%%     email     = "aide@dms.umontreal.ca (Internet)",
%%%     keywords  = "latex, amslatex, ams-latex, theorem",
%%%     abstract  = " Ce fichier est un package conçu pour être
%%%                  utilisé avec la version de LaTeX2e 1995/06/01. Il
%%%                  est prévue pour la classe ``amsbook''. Il en
%%%                  modifie le format des pages, l'entête des
%%%                  sections, etc, afin d'être  conforme au modèle de
%%%                  mémoire de maîtrise de l'Université de
%%%                  Montréal. Finalement ce fichier est grandement
%%%                  inspiré du fichier amsclass.dtx.",
%%%     docstring = "The checksum field contains: CRC-16 checksum,
%%%                  word count, line count, and character count, as
%%%                  produced by Robert Solovay's checksum utility."}
%%%  ====================================================================

%% Pour voir les accents de ce fichier, assurez-vous que votre
%% éditeur de texte lise le fichier en utf-8!

%% La classe <dms> est construite au-dessus de <amsbook>, donc
%% <amsmath>, <amsfonts> et <amsthm> sont automatiquement chargés.
%% Pour un mémoire
\documentclass[12pt,twoside,maitrise]{dms}
%% Pour une thèse
%%\documentclass[12pt,twoside,phd]{dms}

\usepackage[utf8]{inputenc} %Obligatoires
\usepackage[T1]{fontenc}    %

%% <lmodern> incorpore les fontes en T1, pour
%% faciliter le dépôt final. Ceci n'est pas la
%% seule option :
%%  1. Si cm-super est installé, vous pouvez enlever <lmodern>
%%     (à ce moment, la police est un peu plus fidèle
%%      au Computer Modern orginal);
%%  2. Si vous avez une police préférée, par exemple,
%%     <times> ou <euler> ou <mathpazo> (et bien d'autres),
%%     alors vous pouvez remplacer <lmodern> ci-bas.
%% Par contre, si vous faîtes face à un problème d'encapsulation
%% lors dépôt final, il se peut que la solution soit d'utiliser <lmodern>.
%% (Parfois le problème est au niveau de l'installation, donc
%%  essayez de compiler sur un autre ordinateur sur lequel vous êtes
%%  certain·e que l'installation est bonne.)
\usepackage{lmodern}
\DeclareSymbolFont{largesymbols}{OMX}{cmex}{m}{n}

%% Il n'est pas nécessaire d'utiliser <babel>, car
%% les commandes intégrées par la classe <dms>
%% \francais et \anglais font le travail. Néanmoins,
%% certains autres packages nécessitent <babel> (comme
%% <natbib>), donc simplement enlever les % devant <babel>
%% dans ce cas. Attention! Certains packages sont sensibles
%% à l'ordre dans lequel ils sont chargés.
\francais % or
%%\anglais
%%
%%\usepackage[english,frenchb]{babel}

 % ENGLISH OPTION
 % If you call \anglais here before the \begin{document},
 % all the chater's header will be in english, even if you
 % call \francais. To change this, use
 % \entetedynamique

%% La commande \sloppy peut avoir des effets étranges sur les
%% lignes de certains paragraphes.  Dans ce cas, essayez \fussy
%% qui suppresse les effets de \sloppy.
%% (\fussy est normalement le comportement par défaut.)
%% On redéfinit \sloppy, pour tenter de réduire les comportements
%% étranges. Le seul changement apporté à la version originale
%% est la valeur de \tolerance.
\def\sloppy{%
  \tolerance 500%  %9999 dans LaTeX ordinaire, mauvaise idée.
  \emergencystretch 3em%
  \hfuzz .5pt
  \vfuzz\hfuzz}
\sloppy   %appel de \sloppy pour le document
%%\fussy  %ou \fussy

%% Packages utiles.
\usepackage{graphicx,amssymb,subfigure,icomma}
%% icomma       permet d'écrire les nombres décimaux en
%%                  français (p.ex. 1,23 plutôt que 1.23)
%% subfigure    simplifie l'inclusion de figures côtes-à-côtes

%% Packages parfois utiles.
%%\usepackage{dsfont,mathrsfs,color,url,verbatim,booktabs}
%% dsfont       symboles mathématiques \mathds
%% mathrsfs     plus de symboles mathématiques \mathscr
%% color        pour utiliser des couleurs (comparer avec <xcolor>)
%% url          permet l'écriture d'url
%% verbatim     pour écrire du code ou du texte tel quel
%% booktabs     plus de macros pour faire les tableaux
%%                  (voir documentation du package)

%% pour que la largeur de la légende des figures soit = \textwidth
\usepackage[labelfont=bf, width=\linewidth]{caption}

%% les 3 lignes suivante servent à l'affichage de l'index
%% dans le visionneur de pdf. <hyperref> et <bookmark>
%% devraient être les dernier package a être chargé,
%% donc chargez vos packages avant.
\usepackage{hyperref}  % Ajoute les hyperlien
\hypersetup{colorlinks=true,allcolors=black}
\usepackage{hypcap}   % Corrige la position du lien pour les images
\usepackage{bookmark} % Remédie à des petits problème
                      % de <hyperref> (important qu'il
                      % apparaisse APRÈS <hyperref>)

  % Enlever les commentaires du prochaine \hypersetup et
  % le remplir avec l'information pertinente.
  % Ceci ajoute des « méta-données » au pdf.  C'est optionnel,
  % mais recommandé. Vous pouvez voir ces méta-données en
  % ouvrant un visionneur de pdf et en cherchant les propriétés
  % du pdf. (Vous pouvez aussi tapez ' pdfinfo <nom-du-pdf> '
  % dans un terminal.) Ces données sont utiles, par exemple,
  % pour augmenter les chances qu'un algorithme de recherche
  % trouve votre document sur Internet, une fois diffusé.
%%\hypersetup{
%%  pdftitle = {Titre de la thèse / du mémoire},
%%  pdfauthor = {auteur·e},
%%  pdfsubject = {Ex: Transformation de Fourier ; régressions linéaires ; ... },
%%  pdfkeywords = {Ex: mathématiques, statistiques, groupes, variables aléatoires,...}
%%}

%% Définition des environnements utiles pour un mémoire scientifique.
%% La numérotation est laissée à la discrétion de l'auteur·e. L'exemple
%% illustré ici produit « Définition x.y.z »
%%   x = no. chapitre
%%   y = no. section
%%   z = no. définition
%% et la numérotation des corollaires, définitions, etc. se fait
%% successivement.
%%
%% Les macros \<type>name sont telles qu'ils suivent
%% la langue actuelle. (P.ex. si \francais est utilisé,
%% alors \begin{theo} va faire un Théorème et si \anglais
%% est utilisé, \begin{theo} fera un Theorem.)
%%
\newtheorem{cor}{\corollaryname}[section]
\newtheorem{theo}[cor]{\theoremname}
\newtheorem{prop}[cor]{Proposition}
\newtheorem{lem}[cor]{\lemmaname}
\theoremstyle{definition}
\newtheorem{deff}[cor]{\definitionname}
\newtheorem{ex}[cor]{\examplename}
\newtheorem{rem}[cor]{\remarkname}
\newtheorem{algo}[cor]{\algoname}
%% NOTE : Il peut être commode de redéfinir \the<type> pour
%% obtenir la numérotation désirée. Par exemple, pour
%% que les corollaires soit numérotés #section.#sous-section.#sous-sous-section.#paragraphe.#corollaire,
%% on fait
%% \renewcommand\thecor{\theparagraph.\arabic{cor}}

%%%
%%% Si vous préférez que les corollaires, définitions, théorèmes,
%%% etc. soient numérotés séparément, utilisez plutôt un bloc de
%%% commandes de la forme :
%%%

%%\newtheorem{cor}{\corollaryname}[section]
%%\newtheorem{deff}{\definitionname}[section]
%%\newtheorem{ex}{\examplename}[section]
%%\newtheorem{lem}{\lemmaname}[section]
%%\newtheorem{prop}{Proposition}[section]
%%\newtheorem{rem}{\remarkname}[section]
%%\newtheorem{theo}{\theoremname}[section]

%%
%% Numérotation des équations par section
%% et des  tableaux et figures par chapitre.
%% Ceci peut être modifié selon les préférences de l'utilisateur.
\numberwithin{equation}{section}
\numberwithin{table}{chapter}
\numberwithin{figure}{chapter}

%%
%% Si on veut faire un index, il faut décommenter la ligne
%% suivante. Ajouter des mots à l'index avec la commande \index{mot cle} au
%% fur et à mesure dans le texte.  Compiler, puis taper la commande
%% makeindex pour creer les indexs.  Après une nouvelle compilation,
%% vous aurez votre index.
%%

%%\makeindex

%% Il est obligatoire d'écrire à double interligne
%% ou à interligne et demi. On peut soit utiliser
%% le package <setspace> ou \baselinestretch.
%% Le package a tendance a créé des grands blancs,
%% le gabarit décourage son utilisation, mais il en
%% reste à la discrétion de l'utilisateur·e.
%% \usepackage[onehalfspacing]{setspace}
 % ou
\renewcommand{\baselinestretch}{1.286} %Interligne et demi (environ 18pt (12pt+6pt) entre les lignes)

%%%%%%%%%%%%%%%%%%%%%%%%%%%%%%%%%%%%%%%%%%%%%%%%%%%%%%%%%%%%
%%%%%%%%%%%%%%%%%%%%%%%%%%%%%%%%%%%%%%%%%%%%%%%%%%%%%%%%%%%%
%%%%%%%%%%                                     %%%%%%%%%%%%%
%%%%%%%%%% D é b u t    d u    d o c u m e n t %%%%%%%%%%%%%
%%%%%%%%%%                                     %%%%%%%%%%%%%
%%%%%%%%%%%%%%%%%%%%%%%%%%%%%%%%%%%%%%%%%%%%%%%%%%%%%%%%%%%%
%%%%%%%%%%%%%%%%%%%%%%%%%%%%%%%%%%%%%%%%%%%%%%%%%%%%%%%%%%%%
\begin{document}

%%
%% Voici des options pour annoter les différentes versions de votre
%% mémoire. La commande \brouillon imprime, au bas de chacune des pages, la
%% date ainsi que l'heure de la dernière compilation de votre fichier.
%%
%%\brouillon
%%
%%
%% \version est la version de votre manuscrit
%%
\version{1}
\pagenumbering{roman}

%%------------------------------------------------- %
%%              pages i et ii                       %
%%------------------------------------------------- %

%%%
%%% Voici les variables à définir pour les deux premières pages de votre
%%% mémoire.
%%%

\title{Préentraînement d'un modèle ELECTRA}

\author{Simon Théorêt}

\copyrightyear{2025}

\department{Département d'informatique et de recherche opérationnelle}

\date{\today} %Date du DÉPÔT INITIAL (ou du 2e dépôt s'il y a corrections majeures)

\sujet{informatique, \orientation{apprentissage automatique}}
%%\orientation{orientation}%Ce champ est optionnel
%%
%% Voici les disciplines possibles (voir avec votre directeur):
%% \sujet{statistique},
%% \sujet{mathématiques}, \orientation{mathématiques appliquées},
%% \orientation{mathématiques fondamentales}
%% \orientation{mathématiques de l'ingénieur} et
%% \orientation{mathématiques appliquées}

\president{Nom du président du jury}

\directeur{Nom du directeur de recherche}

%%\codirecteur{Nom du 1er codirecteur}         % s'il y a lieu
%%\codirecteurs{Nom du 2e codirecteur}         % s'il y a lieu

\membrejury{Nom du membre de jury}

%%\examinateur{Nom de l'examinateur externe}   %obligatoire pour la these

%% \membresjury{Deuxième membre du jury}  % s'il y a lieu

%%  \plusmembresjury{Troisième membre du jury}    % s'il y a lieu

% Cette option existe encore, mais elle n'a plus sa place
% dans la page titre. L'utiliser seulement si le directeur
% insiste...
%%\repdoyen{Nom du représentant du doyen} %(thèse seulement)

%%
%% Fin des variables à définir. La commande \maketitle créera votre
%% page titre.

%% Pour mettre bouton qui mène à la page titre
%% dans le visionneur de pdf. Peut être enlever.
\pdfbookmark[chapter]{Couverture}{PageUn}

\maketitle

% Pour générer la deuxième page titre, il faut appeler à nouveau \maketitle
% Cette page est obligatoire.
\maketitle

%%------------------------------------------------- %
%%              pages iii                           %
%%------------------------------------------------- %

\francais

\chapter*{Résumé}

...sommaire et mots clés en français...

%%------------------------------------------------- %
%%              pages iv                            %
%%------------------------------------------------- %

\anglais
\chapter*{Abstract}

...summary and keywords in english...

%%------------------------------------------------- %
%%        page v --- Table de matieres              %
%%------------------------------------------------- %

% Pour un mémoire en anglais, changer pour
% \anglais. Noter qu'il faut une permission
% pour écrire son mémoire en anglais.
%%\anglais
\francais
% \cleardoublepage termine la page actuel et force TeX
% a poussé les éléments flottant (fig., tables, etc.) sur
% la page (normalement TeX les garde en suspend jusqu'à ce
% qu'il trouve un endroit approprié). Avec l'option <twoside>,
% la commande s'assure que la prochaine page de texte est sur
% le recto, pour l'impression. On l'utilise ici
% pour que TeX sache que la table des matières etc. soit
% sur la page qui suit.
%% TABLE DES MATIÈRES
\cleardoublepage
\pdfbookmark[chapter]{\contentsname}{toc}  % Crée un bouton sur
% la bar de navigation
\tableofcontents
% LISTE DES TABLES
\cleardoublepage
\phantomsection  % Crée une section invisible (utile pour les hyperliens)
\listoftables
% LISTE DES FIGURES
\cleardoublepage
\phantomsection
\listoffigures

%%%%%%%%%%%%%%%%%%%%%%%%%%%%%%%%%%%%%
%% LISTE DES SIGLES ET ABRÉVIATION %
%%%%%%%%%%%%%%%%%%%%%%%%%%%%%%%%%%%%%
%% Il est obligatoire, selon les directives de la FESP,
%% pour une thèse ou un mémoire d'avoir une liste des sigles et
%% des abréviations.  Si vous considérez que de telles listes ne seraient pas
%% pertinentes (si, par exemple, vous n'utilisez aucun sigle ou abré.), son
%% inclusion ou omission est laissé à votre discrétion.  En cas de doute,
%% parlez-en à votre directeur de recherche, le coadministrateur ou au/à la
%% bibliothécaire.
%%
%% Le gabarit inclut un exemple d'une liste « fait à la main ».  Il existe des outils
%% plus sophistiqués si vous devez inclure une multitude de sigles et abréviations.
%% Par exemple, le package <glossaries> peut faire des index élaborés.  Comme
%% son utilisation est technique, il n'y a pas d'exemple directement dans ce gabarit.
%% On invite les gens qui aurait à l'utiliser à lire la documentation officielle,
%% soit en allant sur https://www.ctan.org/, soit en tapant dans un terminal :
%%
%% texdoc glossaries
%%

\chapter*{Liste des sigles et des abréviations}
\begin{twocolumnlist}{.2\textwidth}{.7\textwidth}
	MLM & Modélisation de langage avec masque, de l'anglais
	\textit{Masked Language Modeling}\\
	TAL & Traitement automatique du langage\\
	NER & Reconnaissance d'entitées, de l'anglais
	\textit{Named-Entity Recognition}\\
\end{twocolumnlist}
% Option de colonnes: definir \colun ou \coldeux
%%% Exemple
%%% \def\colun{\bf} % Première colonne en gras
%%% Pour numéroté les entrées, on peut faire
%%% \newcount\abbrlist
%%% \abbrlist=0
%%% \def\plusun{\global\advance\abbrlist by 1\relax}
%%% \def\colun{\plusun\the\abbrlist. }
%%\def\coldeux{\relax}
%% L'environnement <threecolumnlist> existe aussi pour trois colonnes.

%%------------------------------------------------- %
%%              pages vi                            %
%%------------------------------------------------- %

\chapter*{Remerciements}

Je tiens remercier Joss, pour sa précieuse aide tout au long de mon stage. Je
n'aurais pas pu demander un meilleur superviseur. \\
%TODO: En rajouter un ti peu
Je remercie aussi Momo pour son support moral constant.

%
% Fin des pages liminaires.  À partir d'ici, les
% premières pages des chapitres ne doivent pas
% être numérotées
%

\NoChapterPageNumber
\cleardoublepage
\pagenumbering{arabic}

%%%%%%%%%%%%%%%%%%%%%%%%%%%%%%%%%%%%%%%%%%%%%%%%%%%%%
%%                                                  %
%%   TEXTE DU MÉMOIRE :  introduction page 1,...    %
%%                                                  %
%%%%%%%%%%%%%%%%%%%%%%%%%%%%%%%%%%%%%%%%%%%%%%%%%%%%%

\chapter*{Introduction}
Le domaine du traitement automatique des langues connaît une explosion
fulgurante de techniques, de jeux de données et de modèles permettant de
résoudre de nouveaux problèmes. Néanmoins, bon nombre de ces applications
restent hors de portée des organisations désirant mettre en application des
outils d'apprentissages automatique. En effet, la plupart des modèles de
langues récents sont préentraînés sur des corpus majoritairement anglophones,
avec des jetoniseurs spécialisés pour traiter le contenu anglophone. Ces deux
facteurs limitent les modèles préentraînés disponibles ainsi que leur
performance sur des tâches avec un corpus non anglophone.\\

Druide Inc. est une compagnie basée à Montréal dont le principal produit est
Antidote, un logiciel de correction orthographique et grammaticale. Leur
logiciel phare fait déjà usage de l'apprentissage profond pour leur moteur de
correction en anglais, en plus d'utiliser un correcteur symbolique pour
certains types d'erreurs. Le modèle utilisé en production pour la correction en
anglais fait près de 2 corrections sur 3 et représente une part importante du
moteur de correction. L'équipe de Druide désire mettre en place un modèle de
correction similaire, mais adapté à la langue française. En particulier, ils
désirent préentraîner un modèle ELECTRA avec un corpus et un jetoniseur
francophones pour que le modèle puisse détecter les erreurs grammaticales
présentes dans les textes des utilisateurs d'Antidote.\\

Pour la réalisation du projet, nous disposons d'un jeu de données d'environ 40
GB de données non structuré. De plus, l'entraînement du modèle se fait
localement sur une machine ayant accès à 3 NVIDIA RTX A4000, disposant chacune
de 16 GB de mémoire VRAM.


%%------------------------------------------------- %
%%                pages 1                           %
%%------------------------------------------------- %

\chapter{Druide et ELECTRA}
L'équipe de Druide dispose de deux modèles déjà en place pour la correction des
erreurs. Néanmoins, leur modèle en anglais corrige une plus grande gamme
d'erreurs. Druide désire améliorer leur moteur de correction en français à
l'aide de l'apprentissage profond. %NOTE: Nécessaire?

\section{Contraintes}
Le modèle doit être intégré dans le logiciel principal de Druide, Antidote. Or,
le logiciel Antidote est déployé sur les ordinateurs personnels des usagers.
Cela implique d'importantes contraintes quant aux ressources disponibles pour
l'exécution du modèle, notamment en ce qui à trait à la consommation de
mémoire. De plus, le logiciel Antidote se doit d'être rapide, puisque attendre
plusieurs minutes pour la correction d'un texte volumineux dégrade la qualité
de l'expérience des utilisateurs. En d'autres mots, le modèle doit être rapide
durant l'inférence. Finalement, le logiciel antidote cible deux système
d'exploitation: Windows et MacOS.\@ Le déploiement du modèle sur les machines
des usagers se fait à l'aide des librairies ONNX\cite{onnxruntime} et CoreML.\@
Il est donc nécessaire que le modèle soit supporté par les deux librairies.
En résumé, nous avons des limites quant aux ressources disponibles durant
l'inférence ainsi que des contraintes quant aux couches et modèles utilisables.\\

Ces contraintes ont poussé l'équipe du TAL de Druide à sélectionner
des petits modèles Transformers\cite{vaswani2023attentionneed} avec
encodeur. Ces derniers contiennent environ 14 millions de paramètres.

\section{Méthode de préentraînement ELECTRA}

La méthode ELECTRA\cite{clark2020electrapretrainingtextencoders} est une
méthode inspirée de la modélisation de langage avec masque (Masked Language
Modeling; MLM), mais qui se veut plus efficace et rapide que le MLM. La méthode
ELECTRA consiste à entraîner deux modèles: un petit modèle, appelé le
générateur, et le modèle final, appelé le discriminant. Le générateur reçoit
des jetons masqués et doit prédire quel était le jeton original situé à la
position du masque. Les prédictions du modèle sont échantillonnés, de façon à
obtenir une nouvelle séquence, potentiellement différente de la séquence
originale. Le discriminant reçoit la nouvelle séquence et à pour tâche de
prédire quels jetons sont corrompus et lesquels n'ont pas été modifiés par le
générateur. Seul le discriminant est réutilisé pour l'affinage. La méthode est
visualisée dans la figure %TODO: Insérer figure d'entraînement de ELECTRA ici!\\

Trois éléments rendent l'entraînement du discriminant plus facile.
Premièrement, le générateur dispose de significativement moins de capacité que
le discriminant. En effet, ce dernier contient en général 3 à 4 fois plus de
paramètres (en excluant les couches de projections \textit{embeddings}) que le
générateur. De plus, les entrées du discriminant sont échantillonnées depuis la
distribution engendrée par le générateur, au lieu de sélectionner les entrées
les plus probables selon la distribution du générateur. Finalement, les poids
du générateur sont initialisés aléatoirement et ce dernier est entraîné en même
temps que le discriminant, rendant la tâche de plus en plus difficile au fur et
à mesure que le générateur s'entraîne. Ces trois facteurs rendent la tâche du
discriminant plus facile et permettent de générer des erreurs similaire à ce
que le modèle rencontrera en production.\\

La méthode ELECTRA a été choisie pour deux raisons: c'est une méthode de
préentraînement similaire à la correction d'erreurs dans un texte et la méthode
ELECTRA permet d'augmenter l'efficacité du préentraînement en atteignant des
performances similaires aux performances du MLM en moins d'itérations.

\section{Affinage pour la détection d'erreurs}
Une fois le modèle ELECTRA préentraîné, il est nécessaire d'adapter le modèle
pour que celui-ci soit en mesure de détecter efficacement les erreurs dans les
textes des utilisateurs. Druide a développé une liste des différents types
d'erreurs, permettant de classifier les différents types d'erreurs en de
grandes catégories, telles que les erreurs de virgules, les erreurs de mots
manquants, les erreurs d'accord du nom, etc. Cette liste contient 750
différents types d'erreurs. Chaque erreur fait partie d'une de ces grandes
catégories, et bon nombre de ces erreurs ont une sous-catégorie, précisant
encore plus le contexte associée à l'erreur. La détection d'erreur est
modélisée comme une tâche de détection d'entité nommée (DEN/NER), dans laquelle
chaque jeton dispose d'un classe. Les classes d'erreurs sont représenté avec un
identifiant, tandis que la classe représentant l'absence d'erreurs est
représenté par l'identifiant $O$. Le modèle à comme objectif de spécifier la
classe de chaque jeton de la séquence. Le schéma $IOB2$ \cite{schemas} est
utilisé pour représenter sans ambiguïté les jetons contigus contenus dans la
même erreur.

\section{Infrastructures en place}
Notre tâche principale consistait à préentraîner un modèle ELECTRA.\@ Or, un
modèle ELECTRA est déjà utilisé pour la tâche de correction en anglais. Ce
dernier n'a pas été préentraîné par Druide. En effet, la librairie
Transformers\cite{wolf-etal-2020-transformers} permet un usage libre de
différents modèles ELECTRA préentraînés. De plus, il existe quelques modèles
ELECTRA préentraînés sur des corpus francophone. Cependant, aucun d'entre eux
ne respectent nos contraintes de tailles et de vitesse. Il est donc nécessaire
d'entraîner un modèle à partir d'un initialisation aléatoire.\\

Nous disposons de deux corpus déjà préparés pour préentraîner et affiner un
modèle Electra. Le corpus de préentraînement est une collection de textes non
structuré provenant de nombreuses sources, notamment des manuels, des articles
de blogues, des livres. Ce corpus de préentraînement est appelé corpus des
Combis et représente 40 gigaoctes (Go) de données et 7 milliard de jetons.
C'est un corpus deux fois plus grand que le corpus de préentraînement utilisé
pour le préentraînement par Google du modèle ELECTRA de même taille. Pour
l'affinage, Druide dispose d'un corpus contenant près de 100000 annotations sur
des textes francophones. Ces annotation sont fournies par Druide et proviennent
d'équipes de linguistes et d'annotateurs corrigeant des textes et classifiant
les erreurs qu'ils y trouvent en fonction des types d'erreurs proposés par
Druide.
%TODO: Insérer un exemple d'annotation


\chapter{Préentraînement d'un modèle initial}
Le préentraînement d'un modèle de langue se fait en trois étapes. Il est
nécessaire de prétraiter les données, de sélectionner un jetonniseur adapté à
la tâche ainsi que d'entraîner le modèle sur la tâche de préentraînement.

\section{Normalisation des données et entraînement d'un jetoniseur}
La normalisation consiste à réduire le nombre de charactèrs différents contenus
dans le corpus. C'est une étape importante puisqu'elle permet de réduire la
taille du vocabulaire du jetoniseur sans pour autant perdre des éléments
syntaxiques. Notre normalisation consistait à tranformer tous les espaces en le
même charactère d'espace (espaces insécables, tabulations), de transformer tous
les guillements (guillements français, guillemets informatiques, etc.) en
guillemets anglais, de retirer les espaces en trop et modifier les types
d'apostrophes pour que ceux-ci soient uniformes. La normalisation modifie aussi
certains les espacements entre certains mots, de façon à ce que par exemple le
texte "11 ème étage" devienne "11ème étage".\\% TODO: Retravailler cette phrase

Une fois le texte normalisé, il est possible d'entraîner un jetoniseur adapté à
la tâche. En l'occurence, nous avons sélectionner le jetoniseur Wordpiece.
%TODO: Ajouter citation pour wordpiece.
C'est le jetoniseur choisi par les auteurs de l'article de ELECTRA et est
actuellement utilisé en production chez Druide. Pour l'entraînement du
jetoniseur, nous utilisons le corpus des combis normalisé, comprenant environ
40 GO de données. Les hyper-paramètres sélectionnés pour le jetoniseur sont
   dcnnés dans la figure %TODO: Insérer fig.

\chapter{Modèle final}
\chapter{Conclusions}
%TODO: Discuter des directions futures: Génération de données artificielles
%%--------------%
%%     index    %
%%--------------%

%% S'il y a lieu, décommenter la ligne pour mettre votre index

%%\printindex

%%------------------------------------------------- %
%%         références --- bibliographie             %
%%------------------------------------------------- %
% Enlever les commentaires de la prochaine commande si vous préférez que le
% chapitre s'appelle « Références » plutôt que « Bibliographie » (au choix selon le contexte).
%%\let\bibname=\refname

%% Lorsque vous serez prêt à faire afficher votre bibliographie
%% et vos références, enlevez les commandaires des commandes suivantes
%% et donnez le nom de votre fichier .bib à la commande \bibliography{..}
%% (consultez l'exemple au besoin).  Vous pouvez utiliser le style de votre
%% choix.
\bibliographystyle{amsplain-french}     % Le style de la bibliographie. Notons que
% les extensions ne sont pas données pour ces deux fichiers.
\def\bibname{R\'ef\'erences} % Nom obligatoire de la section des références.
% On utilise \'e car le é cause des problèmes
% dans la table des matière
%% ENGLISH
%\def\bibname{References}
\bibliography{ref}     % La base de données contenant des entrées bibliographiques.
% Seules celles référencées dans le texte seront ajoutées
% \`a la bibliographie.

%%------------------------------------------------- %
%%                  Annexe A                        %
%%------------------------------------------------- %

\appendix
\chapter{Le titre}

\section{Section un de l'Annexe A}

...texte...

\chapter{Les différentes parties et leur ordre d'apparition}

J'ajoute ici les différentes parties d'un mémoire ou d'une thèse ainsi
que leur ordre d'apparition tel que décrit dans le guide de
présentation des mémoires et des thèses de la Faculté des études
supérieures.  Pour plus d'information, consultez le guide sur le site
web de la facutlé (www.fes.umontreal.ca).

\newcount\colnum
\colnum=1
\def\i{\number\colnum. \global\advance\colnum by 1\ignorespaces}
\begin{table}[p]
	\begin{center}
		\begin{tabular}{|l|l|r|}\hline
			\noindent\hfil
			\textbf{\strut Ordre des éléments constitutifs du mémoire ou de la thèse}
			\hfil\span\omit\span\omit                                                   \\\hline % \span\omit pour couvrir plus d'une
			% case sans utiliser le package multirow ou autre
			\i & La page de titre                                        & obligatoire  \\\hline
			\i & La page d'identification des membres du jury            & obligatoire  \\\hline
			\i & Le résumé en français et les mots clés français\kern3em & obligatoires \\\hline
			\i & Le résumé en anglais et les mots clés anglais           & obligatoires \\\hline
			\i & Le résumé dans une autre langue que l'anglais           & obligatoire  \\
			   & ou le français (si le document est écrit dans           &              \\
			   & une autre langue que l'anglais ou le français)          &              \\\hline
			\i & Le résumé de vulgarisation                              & facultatif   \\\hline
			\i & La table des matières, la liste des tableaux,           & obligatoires \\
			   & la liste des figures ou autre                           &              \\\hline
			\i & La liste des sigles et des abréviations                 & obligatoire  \\\hline
			\i & La dédicace                                             & facultative  \\\hline
			\i & Les remerciements                                       & facultatifs  \\\hline
			\i & L'avant-propos                                          & facultatif   \\\hline
			\i & Le corps de l'ouvrage                                   & obligatoire  \\\hline
			\i & Les index                                               & facultatif   \\\hline
			\i & Les références bibliographiques                         & obligatoires \\\hline
			\i & Les annexes                                             & facultatifs  \\\hline
			\i & Les documents spéciaux                                  & facultatifs  \\\hline
		\end{tabular}
	\end{center}
\end{table}

\end{document}



\endinput
%%
%% End of file `gabaritmem.tex'.
